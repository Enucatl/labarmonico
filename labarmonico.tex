\documentclass[italian,a4paper,10pt]{article}
\usepackage{babel,geometry,dialogue}
\usepackage[utf8]{inputenc}
\usepackage[T1]{fontenc}
\usepackage{ae,aecompl}
\frenchspacing
%
%------------- impostazioni pacchetto geometry
%
%----larghezza del testo
\geometry{textwidth=.65\paperwidth,
%
%----altezza del testo
textheight=50\baselineskip }
%
%------------- ridefinizione simbolo per elenchi puntati: en dash
\renewcommand{\labelitemi}{\textbf{--}}
\newcommand{\ac}{\speak{Achille}}
\newcommand{\ta}{\speak{Tartaruga}}
\newcommand{\gen}{\speak{Genio}}
\title{\textbf{Piccolo labirinto armonico}\\
\normalsize Tratto da \emph{G\"{o}del, Escher, Bach: un'Eterna Ghirlanda Brillante}}
\date{}
\author{Douglas R. Hofstadter}
\begin{document}
\pagestyle{plain}
\maketitle
%------------------------------------------------
[Si tratta di un dialogo tra Achille e la Tartaruga che, in una delle loro peripezie, trovano una Lampada. Il dialogo vuole illustrare i ragionamenti ricorsivi e annidati, ma ci sono allusioni anche al sistema dei numeri naturali e altre ancora più sottili, che vengono svelate solo più avanti nel libro.]
\begin{dialogue}
\ac Me n'ero dimenticato, signorina T. Ho questa lampada magica! Ma che cos'ha di magico?
\ta Oh, la solita roba: un genio.
\ac Cosa? Vuol dire che, quando la si strofina, appare un genio che esaudisce i desideri?
\ta Proprio così. Che cosa voleva, la manna dal cielo?
\ac \`E fantastico! Io posso esprimere un qualsiasi desiderio, vero? Ho sempre desiderato che un giorno mi accadesse una cosa simile\dots
\direct{E così Achille strofina delicatamente la grande lettera `L' incisa sulla superficie di rame della lampada\dots Immediatamente si sprigiona un'enorma nuvola di fumo, e in essa i cinque amici vedono prender forma uno strano fantasma che torreggia sopra di loro.}
\gen Salve, amici miei, e grazie per aver recuperato la mia lampada dal malefico Duo delle Lucertole.
\direct{Così dicendo, il Genio prende la Lampada e la nasconde tra le pieghe della sua veste che si srotola a spirale dalla Lampada.}
Come segno di gratitudine per il vostro eroico gesto, vorrei offrirvi da parte della mia Lampada la possibilità di esaudire tre vostri desideri.
\ac Stupefacente! Non le sembra, signorina T.?
\ta Certamente. E ora, avanti, Achille, esprima il primo desiderio.
\ac Uh! Ma che cosa potrei desiderare? Ah, lo so: quello che pensai la prima volta che lessi \emph{Le mille e una notte} (quella raccolta di novelle insensate (e a scatole cinesi)). Desidero poter esprimere \textsc{cento} desideri invece di tre! Molto astuto, vero, signorina T.? Scommetto che \textsc{lei} non avrebbe mai pensato a un trucco simile. Mi sono sempre chiesto perché quegli sciocchi delle novelle non ci hanno mai provato.
\ta Forse ora scoprirà da se la risposta.
\gen Mi dispiace, Achille, ma non posso esaudire meta-desideri.
\ac Desidero sapere che cos'è un ``meta-desiderio''!
\gen Ma \textsc{questo} è un meta-meta-desiderio, Achille, e io non posso esaudire neanche questi.
\ac Cooosa? Nn riesco proprio a capire.
\ta Perché non cambia i termini della sua ultima richiesta, Achille?
\ac Che cosa intende dire? Perché dovrei farlo?
\ta Vede, lei ha cominciato col dire: ``desidero''. Dato che vuole soltanto un'informazione, perché non fa semplicemenete una domanda?
\ac Va bene, per quanto non ne veda il motivo. Dimmi, Genio, che cos'è un meta-desiderio?
\gen \`E semplicemente un desiderio riguardante altri desideri. E io non sono autorizzato ad esaudire meta-desideri. La mia competenza è limitata soltanto a desideri comuni, come avere dieci bottiglie di birra, incontrare Miss Universo a quattr'occhi, o vincere un viaggio per due a Copacabana. Cose semplici come queste. Ma i meta-desideri non posso esaudirli. Il SIGNOR non lo permette.
\ac Il SIGNOR? Chi è il SIGNOR? E perché non ti permetterebbe di esaudire i meta-desideri? Non mi sembrano poi una gran cosa rispetto ai desideri che hai nominato.
\gen Be', vedi, è una faccenda complicata. Perché non lasci perdere ed esprimi i tre desideri? O uno almeno, non ho mica tanto tempo da perdere\dots
\ac Oh, sono avvilito! \textsc{Speravo veramente} di poter esprimere il desiderio di poter esprimere cento desideri\dots
\gen Non sopporto di vedere la gente soffrire in questo modo. E inoltre i meta-desideri sono i miei desideri preferiti. Vediamo se posso fare qualcosa. Mi ci vorrà un secondo\dots
\direct{Il Genio estrae dalle pieghe evanescenti della sua veste un oggetto che assomiglia alla Lampada di rame che aveva riposto, con la sola differenza che questa è d'argento; e dove l'altra aveva incisa una `L', questa reca la scritta `ML' in lettere più piccole in modo da occupare lo stesso spazio.}
\ac Che cos'è quella?
\gen Questa è la mia Meta-Lampada\dots
\direct{Strofina la Meta-Lampada; immediatamente si sprigiona un'enorme nuvola di fumo, e in essa tutti vedono prender forma uno strano fantasma che torreggia sopra di loro.}
\speak{Meta-Genio} Io sono il Meta-Genio. Mi hai chiamato, o Genio? Qual è il tuo desiderio?
\gen Ho un desiderio speciale da esprimere a te, o Genìde, e al SIGNOR. Desidero la temporanea sospensione di tutte le restrizioni di tipo riguardanti i desideri, per la durata di un Desiderio senza Tipo. Puoi esaudire questo desiderio per me?
\speak{Meta-Genio} Devo inoltrarlo attraverso i soliti canali, naturalmente. Mezzo secondo, per favore.
\direct{E due volte più velocemente del Genio, questo Meta-Genio estrae dalle pieghe evanescenti della sua veste un oggetto che assomiglia alla Meta-Lampada d'argento, con la sola differenza che questa è d'oro; e dove la precedente recava la scritta `ML', questa porta inciso `MML' in lettere più piccole in modo da occupare lo stesso spazio.}
\ac E questa cos'è?
\speak{Meta-Genio} Questa è la mia Meta-Meta-Lampada\dots
\direct{Strofina la Meta-Meta-Lampada; immediatamente si sprigiona un'enorme nuvola di fumo, e in essa tutti vedono prender forma uno strano fantasma che torreggia sopra di loro.}
\speak{Meta-Meta-Genio} Io sono il Meta-Meta-Genio. Mi hai chiamato, o Meta-Genio? Qual è il tuo desiderio?
\speak{Meta-Genio} Ho un desiderio speciale da esprimere a te, o Genìde, e al SIGNOR. Desidero la temporanea sospensione di tutte le restrizioni di tipo riguardanti i desideri, per la durata di un Desiderio senza Tipo. Puoi esaudire questo desiderio per me?
\speak{Meta-Meta-Genio} Devo inoltrarlo attraverso i soliti canali, naturalmente. Un quarto di secondo, per favore.
\direct{E con velocità doppia rispetto al Meta-Genio, questo Meta-Meta-Genio estrae dalle pieghe evanescenti della sua veste un oggetto che assomiglia alla Meta-Lampada d'argento, con la sola differenza che questa è fatta di\dots}\\
\begin{center}
$\vdots$ \\
\begin{tiny}[SIGNOR]\end{tiny}\\
$\vdots$ \\\end{center}
\direct{\dots si ritrae nella Meta-Meta-Meta-Lampada, che il Meta-Meta-Genio ripone nella sua veste, mettendoci il doppio del tempo impiegato dal Meta-Meta-Meta-Genio per la stessa operazione.}
Il tuo desiderio è esaudito, o Meta-Genio.
\speak{Meta-Genio} Grazie, o Genìde, e grazie, o SIGNOR.
\direct{\dots Il Meta-Meta-Genio, come tutti i suoi superiori prima di lui, si ritrae nella Meta-Meta-Lampada, che il Meta-Genio ripone nella sua veste, mettendoci il doppio del tempo impiegato dal Meta-Meta-Genio per la stessa operazione.} Il tuo desiderio è esaudito, o Genio.
\gen Grazie, o Genìde, e grazie, o SIGNOR.
\direct{\dots E il Meta-Genio, come tutti i suoi superiori prima di lui, si ritrae nella Meta-Lampada, che il Genio ripone nella sua veste, mettendoci il doppio del tempo impiegato dal Meta-Genio per la stessa operazione.} Il tuo desiderio è esaudito, o Achille. \direct{Ed è passato precisamente un secondo da quando ha detto: ``Mi ci vorrà un secondo''}
\ac Grazie, o Genìde, e grazie, o SIGNOR. 
\gen Sono contento di informarti, Achille, che puoi esprimere esattamente un (1) Desiderio senza Tipo, vale a dire un desiderio, o un meta-desiderio, o un meta-meta-desiderio, con tanti ``meta'' quanti ne desideri, anche in numero infinito (se lo desideri).
\ac Oh, grazie infinite, Genio. Ma adesso hai stuzzicato la mia curiosità. Prima di esprimere il mio desiderio, ti dispiace dirmi chi o che cosa è il SIGNOR?
\gen Ma figurati, ``SIGNOR'' è un acronimo che sta per ``SIGNOR Induce Genìdi Nuovi Operando Ricorsivamente''. La parola ``Genìdi'' designa Geni, Meta-Geni, Meta-Meta-Geni e così via. \`E una parola senza Tipo.
\ac Ma\dots Ma\dots come può ``SIGNOR'' essere una parola nel suo stesso acronimo? Questo non ha senso!
\gen Tu non hai alcuna familiarità con acronimi ricorsivi! Io pensavo che tutti li conoscessero. Vedi, ``SIGNOR'' sta per ``SIGNOR Induce Genìdi Nuovi Operando Ricorsivamente'' che può essere esplicitato così: ``SIGNOR Induce Genìdi Nuovi Operando Ricorsivamente, Induce Genìdi Nuovi Operando Ricorsivamente'', che può a sua volta essere esplicitato così: ``SIGNOR Induce Genìdi Nuovi Operando Ricorsivamente, Induce Genìdi Nuovi Operando Ricorsivamente, Induce Genìdi Nuovi Operando Ricorsivamente'' che a sua volta può essere esplicitato\dots si può andare avanti quanto si desidera.
\ac Ma non si finisce mai!
\gen Naturalmente no. Non si può mai esplicitare completamente il SIGNOR.
\ac Hum\dots Che rompicapo. Che cosa intendevi quando hai detto al Meta-Genio ``Ho un desiderio speciale da esprimere a te, o Genìde, e al SIGNOR''?
\gen Io volevo fare una richiesta non soltanto al Meta-Genio, ma a tutti i Genìdi al di sopra di lui. Il metodo dell'acronimo ricorsivo realizza ciò in maniera abbastanza naturale. Vedi, quando il Meta-Genio ha ricevuto la mia richiesta, ha dovuto trasmetterla verso l'alto al suo SIGNOR. Così egli ha inoltrato un messaggio analogo al Meta-Meta-Genio, che a sua volta ha ripetuto l'operazione con il Meta-Meta-Meta-Genio\dots Risalendo lungo questa catena, il messaggio raggiunge il SIGNOR.
\ac Capisco, vuoi dire che il SIGNOR siede in cima alla scala dei genìdi?
\gen No, no, no. Non c'è niente ``in cima'' poiché non c'è una cima. Ecco perché il SIGNOR è un acronimo ricorsivo. SIGNOR non è una specie di genìde finale; il SIGNOR è la torre dei genìdi al di sopra di ogni genìde dato.
\ta Mi sembra che ogni singolo genìde debba avere un concetto diverso del SIGNOR, poiché per ogni genìde il SIGNOR è l'insieme dei genìdi al di sopra di lui e non vi sono due genìdi per i quali questi insiemi coincidano.
\gen Lei ha pienamente ragione. E siccome io, in quanto Genio, sono il genìde infimo, la mia nozione di SIGNOR è la più ampia. Ho compassione per questi genìdi superiori, che credono di essere in qualche modo più vicini al SIGNOR. Quale empietà!
\ac Per tutti i diavoli, ci sarà voluto del genio per inventare il SIGNOR.
\ta Lei crede davvero a tutta questa storia del SIGNOR, Achille?
\ac Oh, certo che ci credo. Perché, lei è atea, oppure agnostica, signorina T.?
\ta Non credo di essere agnostica, forse sono meta-agnostica.
\ac Cooosa? Non riesco proprio a capire.
\ta Vediamo\dots Se io fossi meta-agnostica, avrei dei dubbi sul fatto di essere agnostica o meno, ma io non sono proprio sicura di essere su \textsc{queste} posizioni; quindi, ciò significa che sono meta-meta-agnostica (mi sembra). Ma lasciamo perdere.\newline
[\dots]
\ac Hum\dots questo mi suggerisce qualcosa per il mio desiderio.
\ta Davvero? Che cosa?
\ac Desidero che il mio desiderio non venga esaudito.
\direct{In quel momento accade un evento --- ma è ``evento'' la parola giusta? --- che non si può descrivere, e quindi non sarà fatto alcun tentativo per descriverlo.}
\end{dialogue}

\end{document}
